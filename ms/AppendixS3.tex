%\documentclass[fleqn,10pt]{wlscirep}
%\documentclass[fleqn,10pt]{svjour3}
\documentclass{svjour3}
\usepackage{lineno}
\usepackage{graphicx}
\usepackage{natbib}
\usepackage{setspace}
\usepackage{amsmath}
\usepackage{hyperref}
\usepackage{caption}
\begin{document}
\title{Abundance and distribution of ice-associated seals in the western Bering Sea and Sea of Okhotsk, 2012-2013}

\author{Paul B. Conn \and
        Irina S. Trukhanova \and
        Peter L. Boveng \and
        Alexander N. Vasiliev \and
        Vladimir I. Chernook
}
\institute{P. B. Conn \at Marine Mammal Laboratory, Alaska Fisheries Science Center, NOAA National Marine Fisheries Service, Seattle, WA 98115, USA  \\
  Tel.: +1-206-526-4235 \\
  Fax.: +1-206-526-6615\\
  \email{paul.conn@noaa.gov}
     \and
  I. S. Trukhanova \at Polar Science Center, Applied Physics Laboratory, University of Washington, 1013 NE 40th St., Seattle, WA 98105 USA
     \and
  P. L. Boveng \at Marine Mammal Laboratory, Alaska Fisheries Science Center, NOAA National Marine Fisheries Service, Seattle, WA 98115, USA
     \and
  A. N. Vasiliev \at Autotonomous Non-Commercial Organization, Ecological Center, ECOFACTOR, 11/1-10N Neyshlotskiy Lane, Saint-Petersburg, 194044 Russia
     \and
  V. Chernook \at Autotonomous Non-Commercial Organization, Ecological Center, ECOFACTOR, 11/1-10N Neyshlotskiy Lane, Saint-Petersburg, 194044 Russia
}

\journalname{Polar Biology}
\titlerunning{Online Resource 3: Supplementary Figures}
\authorrunning{P.B. Conn et al.}


\captionsetup{labelformat=empty}

\maketitle

\large
\bigskip
\centerline{\textbf{Online Resource 3}: Supplementary figures}
\bigskip
\small

\pagebreak

\textbf{Online Resource 3A}: Marginal effects of covariates on seal abundance in the Sea of Okhotsk.
\begin{figure}[ht]
\centering
\includegraphics[width=\linewidth]{Cov_effects_Okhotsk}
\caption{\textbf{Fig 3A}. Effects of environmental and physiographic covariates on seal abundance in the Sea of Okhotsk on the scale of the linear predictor (i.e., logit scale). Depth is in meters, ``Ice" represents the proportion of a grid cell with sea ice, and the remaining values represent distance to certain features (which were calculated in projected space and standardized to their mean, thus lacking meaningful units).  In particular, ``$\text{Dist}_\text{edge}$" represents distance to ice edge, ``$\text{Dist}_\text{land}$" represents distance to land, and ``$\text{Dist}_\text{shelf}$" represents distance to the closest 200 m isobath. Covariates are measured at the centroid of each grid cell.   }
\label{fig:covo}
\end{figure}

\pagebreak

\textbf{Online Resource 3B}: Goodness-of-fit diagnostics for 2013 surveys in the Sea of Okhotsk.
\begin{figure}[ht]
\centering
\includegraphics[width=\linewidth]{GOF_Okhotsk}
\caption{\textbf{Fig 3B}. Randomized quantile residual diagnostics for our preferred ``base" model fitted to count data obtained in 2013 aerial surveys in the Sea of Okhotsk.  Each histogram corresponds to counts of spotted seals (``Obs = 1"), ribbon seals (``Obs = 2"), bearded seals (``Obs = 3"), ringed seals (``Obs = 4"), unknown (but photographed) seal species (``Obs = 5"), pups (``Obs = 6"), and unphotographed seals (``Obs = 7").  In a well fitting model, each histogram would have a uniform distribution, an assumption which is tested with a $\chi^2$ discrepancy statistic (p-values in headers).  In particular, we tend to under-predict medium-high counts of spotted seals, high values of ribbon seals, and medium to high values of unknown (but photographed seal species).}
\label{fig:GOFo}
\end{figure}

\pagebreak


\textbf{Online Resource 3C}: Marginal effects of covariates on seal abundance in 2012 in the western Bering Sea.
\begin{figure}[ht]
\centering
\includegraphics[width=\linewidth]{wBS2012_cov_plot}
\caption{\textbf{Fig 3C}. Effects of environmental and physiographic covariates on seal abundance in the western Bering Sea in 2012 on the scale of the linear predictor (i.e., logit scale). ``Ice" represents the proportion of a grid cell with sea ice, and the remaining values represent distance to certain features (which were calculated in projected space and standardized to their mean, thus lacking meaningful units).  In particular, ``$\text{dist}\_\text{edge}$" represents distance to ice edge, ``$\text{dist}\_\text{land}$" represents distance to land, and ``$\text{dist}\_\text{shelf}$" represents distance to the closest 1000 m isobath. Covariates are measured at the centroid of each grid cell.   }
\label{fig:covsWBS2012}
\end{figure}

\pagebreak

\textbf{Online Resource 3D}: Goodness-of-fit diagnostics for 2012 surveys in the western Bering Sea.
\begin{figure}[ht]
\centering
\includegraphics[width=\linewidth]{wBS_GOF_2012}
\caption{\textbf{Fig 3D}. Randomized quantile residual diagnostics for our preferred ``base" model fitted to count data obtained in 2012 aerial surveys in the western Bering Sea.  Each histogram corresponds to counts of spotted seals (``Obs = 1"), ribbon seals (``Obs = 2"), bearded seals (``Obs = 3"), ringed seals (``Obs = 4"), unknown (but photographed) seal species (``Obs = 5"), pups (``Obs = 6"), and unphotographed seals (``Obs = 7").  In a well fitting model, each histogram would have a uniform distribution, an assumption which is tested with a $\chi^2$ discrepancy statistic (p-values in headers).  In particular, we tend to under-predict medium-high counts of spotted seals.}
\label{fig:GOFwBS2012}
\end{figure}

\pagebreak

\textbf{Online Resource 3E}: Marginal effects of covariates on seal abundance in 2013 in the western Bering Sea.
\begin{figure}[ht]
\centering
\includegraphics[width=\linewidth]{wBS2013_cov_plot}
\caption{\textbf{Fig 3E}. Effects of environmental and physiographic covariates on seal abundance in the western Bering Sea in 2013 on the scale of the linear predictor (i.e., logit scale). ``Ice" represents the proportion of a grid cell with sea ice, and the remaining values represent distance to certain features (which were calculated in projected space and standardized to their mean, thus lacking meaningful units).  In particular, ``$\text{dist}\_\text{land}$" represents distance to land, and ``$\text{dist}\_\text{shelf}$" represents distance to the closest 1000 m isobath. Covariates are measured at the centroid of each grid cell.   }
\label{fig:covsWBS2013}
\end{figure}

\pagebreak

\textbf{Online Resource 3F}: Goodness-of-fit diagnostics for 2013 surveys in the western Bering Sea.
\begin{figure}[ht]
\centering
\includegraphics[width=\linewidth]{wBS_GOF_2013}
\caption{\textbf{Fig 3D}. Randomized quantile residual diagnostics for our preferred ``base" model fitted to count data obtained in 2013 aerial surveys in the western Bering Sea.  Each histogram corresponds to counts of spotted seals (``Obs = 1"), ribbon seals (``Obs = 2"), bearded seals (``Obs = 3"), ringed seals (``Obs = 4"), unknown (but photographed) seal species (``Obs = 5"), pups (``Obs = 6"), and unphotographed seals (``Obs = 7").  In a well fitting model, each histogram would have a uniform distribution, an assumption which is tested with a $\chi^2$ discrepancy statistic (p-values in headers).  In particular, we tend to over-predict medium-high counts of ribbon seals and under-predict very large counts of ribbon seals.}
\label{fig:GOFwBS2013}
\end{figure}

\end{document}

